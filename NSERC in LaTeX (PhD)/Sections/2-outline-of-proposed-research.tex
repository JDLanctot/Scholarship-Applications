\begin{outline-of-proposed-research}
\vspace{-1.0em}
\textbf{Introduction:} Networked systems form the backbone of modern infrastructure, from power grids to communication networks. However, these systems are inherently vulnerable to cascading failures\cite{dobson2007complex, bakshy2011everyone} and targeted attacks,\cite{albert2000attack, cohen2001breakdown} posing significant risks to their stability and functionality. With the spectacular advances in Deep Learning over the past decade, humanity has gained new tools to solve previously intractable problems,\cite{mnih_playing_2013, silver_general_2018, moravcik_deepstack_2017, vinyals_grandmaster_2019} including the potential to explore network vulnerabilities.\cite{dai_learning_2017} This research aims to investigate and enhance network robustness through a comprehensive approach combining deep learning techniques with established network science models.

\textbf{Research Context and Significance:} Recent advancements in network science have revealed the complex interplay between network topology and system vulnerability.\cite{DSouza2013} However, there remains a critical gap in our understanding of how to effectively mitigate these vulnerabilities,\cite{brummitt2012suppressing, Bhaumik2013, Vieira2007, buldyrev2010catastrophic, rosato2008modelling, Motter2002} especially in the face of sophisticated attacks or cascading failures. Real-world networks such as power grids are fundamentally vulnerable to attacks on a small number of key components (nodes), whose removal can cause the system as a whole to collapse. While identifying these "Achilles Heels" of a given network is a computationally intractable (NP-Hard) problem, deep learning might effectively surmount this obstacle.\cite{silver_general_2018, moravcik_deepstack_2017, vinyals_grandmaster_2019, dai_learning_2017} This research addresses this gap by integrating cutting-edge deep learning methodologies with classical network models, potentially revolutionizing our approach to network security and resilience.

The significance of this work extends beyond theoretical advancements in network science. By developing novel strategies for network protection and failure mitigation, this research has far-reaching implications for critical infrastructure protection, communication network resilience, and cybersecurity enhancements.

\textbf{Research Objective and Hypothesis:}

\vspace{-1.0em}
\textit{Objective:} Develop and implement an integrated framework that combines deep learning techniques with network science models to enhance the robustness of complex networked systems against both targeted attacks and cascading failures, while also understanding the limits of deep learning's ability to dismantle networks by removing key nodes.

\textit{Hypothesis:} The integration of deep learning methodologies (specifically, Deep Policy Gradient Learning and Reinforcement Learning) with classical network models will significantly outperform traditional approaches in identifying critical nodes for network fragmentation, mitigating cascading failures through targeted interventions, and predicting and managing failure propagation. Furthermore, deep learning's capacity to discover key network features will decrease as more information about the network structure is hidden, and this trend will hold for networks of varying sizes and degrees of interconnection.

\textbf{Methodology:}

\vspace{-1.0em}
\textit{Graph Fragmentation using Deep Reinforcement Learning} - We will create a framework that maps the real-world problem of network attack and defense to a two-player strategy game solvable with Deep Graph Learning. This framework will replicate network attack through the network dismantling effects of node destruction. Neural network embedding frameworks will be employed to capture important network features, transforming the structure of the data into a fixed dimensional tensor suitable for deep reinforcement learning. We will develop a Deep PG-Learning algorithm to identify optimal strategies for graph fragmentation across various network topologies, evaluating it against traditional graph theory methods.

\textit{Cascading Failures using Abelian Sandpile Model (ASM)} - The Abelian Sandpile Model \cite{bak1987, bak1988} will be extended to more complex network structures,\cite{Christensen1993} including scale-free and small-world networks, to better represent real-world systems. We will develop heuristics to optimize mitigation strategies, specifically targeted sand dropping and link rewiring, for reducing the impact of cascading failures. The effectiveness of these strategies will be analyzed in altering the distribution of avalanche sizes, with a particular focus on reducing the frequency and magnitude of large-scale failures.

\textit{Deep Learning Applied to Abelian Sandpile Model} - A reinforcement learning framework will be developed to model and predict the behavior of the Abelian Sandpile Model on complex networks. Neural networks will be trained to accurately predict avalanche sizes and distributions based on initial sandpile configurations and network topologies. We will investigate the capability of deep learning models to identify critical nodes or regions in the network that are prone to triggering large avalanches, and evaluate the scalability and computational efficiency of these approaches compared to traditional simulation methods.

\textbf{Expected Outcomes and Impact:} This research is expected to yield significant theoretical advancements in graph fragmentation theory, cascading failure dynamics, and the application of deep learning to NP-hard problems in network science. We anticipate determining whether deep learning of weak points in a network can be thwarted by strategic concealment of key network information, and what the optimal defensive strategy might be. If an attacker's ability to quickly dismantle networks cannot be reduced in a meaningful way, this might reveal a fundamental weakness of infrastructure and other real-world networks to malicious attacks by a deep-learning equipped adversary.

Practical applications of this work are wide-ranging, including enhanced tools for identifying vulnerabilities in critical infrastructure networks, improved models for predicting and quantifying the risk of large-scale network failures, and guidelines for designing more resilient network structures. These outcomes have potential impacts in fields such as power grid management, transportation systems, and communication networks.

Furthermore, the development of open-source libraries implementing the proposed algorithms and models will facilitate further research and practical applications in the field, contributing to the broader scientific community.

\textbf{Conclusion:} This research represents a significant step forward in our understanding and management of complex networked systems. By integrating advanced machine learning techniques with established network models, we aim to develop more sophisticated, adaptive methods for maintaining network integrity under diverse stress conditions. The insights gained from this work will not only advance the field of network science but also provide practical tools for enhancing the robustness of real-world systems, potentially revolutionizing our approach to network security and resilience. The deliverable from this research will be defensive strategies which thwart malicious deep learning seeking to destroy networks.
\end{outline-of-proposed-research}