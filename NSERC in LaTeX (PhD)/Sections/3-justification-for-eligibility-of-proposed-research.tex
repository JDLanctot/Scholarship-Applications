\begin{justification-for-eligibility-of-proposed-research}
\vspace{-1.0em}
The proposed research project, focusing on network robustness through the application of deep learning techniques and complex systems modeling, falls squarely within the Natural Sciences and Engineering (NSE) domain. While the outcomes of this research may have implications for fields that intersect with health or social sciences, the core objectives, methodologies, and theoretical foundations are firmly rooted in computer science, mathematics, and engineering.

\textbf{NSE Research Challenges}

\vspace{-1.0em}
\textit{Graph Theory and Network Science:} The project's foundation lies in advanced graph theory, a branch of mathematics. We aim to develop novel algorithms for graph fragmentation and explore the topological properties of complex networks. This work extends current understanding in discrete mathematics and theoretical computer science.

\textit{Deep Reinforcement Learning:} A significant portion of the research involves developing and implementing deep reinforcement learning algorithms, specifically Deep PG-Learning. This work is grounded in computer science and artificial intelligence, focusing on the computational challenges of training AI agents to solve NP-hard problems in graph theory.

\textit{Complex Systems Modeling:} The extension of the Abelian Sandpile Model to various network structures involves sophisticated mathematical modeling and simulation techniques. This work primarily advances knowledge in statistical physics and complex systems theory, core areas within NSE.
Computational Efficiency and Scalability:
A key challenge in this research is developing computationally efficient methods for analyzing large-scale networks. This involves algorithm design and optimization, which are fundamental computer science and software engineering problems.

\textbf{NSE Advancement:} The proposed research will primarily advance knowledge in the following NSE areas:

\textit{Network Theory:} Developing new methodologies for understanding and quantifying network robustness.

\textit{Artificial Intelligence:} Advancing the application of deep learning to graph-structured data and complex systems.

\textit{Computational Complexity:} Exploring new approaches to solving NP-hard problems in graph theory using machine learning techniques.

\textit{Data Structures and Algorithms:} Designing novel algorithms for graph manipulation and analysis.

\textit{Applied Mathematics:} Extending mathematical models (like the Abelian Sandpile Model) to more complex network structures.

While the outcomes of this research may have applications in fields such as infrastructure protection or cybersecurity, which could intersect with health or social sciences, the core research questions, methodologies, and intended advancements are firmly within the NSE domain. The project does not aim to study health outcomes, human behavior, or social phenomena, which would fall under CIHR or SSHRC mandates. In conclusion, this research proposal is best aligned with NSERC's mandate due to its focus on advancing fundamental knowledge in computer science, mathematics, and engineering, with the primary goal of developing new computational and mathematical tools for enhancing network robustness.
\end{justification-for-eligibility-of-proposed-research}
