\begin{contributions}
\section*{I - Contributions}
\vspace{-1.5em}
% \textbf{Contributions to research and development}\\
\textbf{a) Articles published or accepted in peer-reviewed journals}

\vspace{-0.6em}
Chuphal, P., \textbf{Lanct\^{o}t, J.D.}, Cornelius, S.P., Brown, A.I. (2024) Mitochondrial network branching enables rapid protein spread with slower mitochondrial dynamics. PRX Life. (Masters work)

Ramey, H.L., Rayner, M., Mahdy, S.S.,  Lawford, H.L., \textbf{Lanct\^{o}t, J.D.}, Campbell, M., Valenzuela, E., Miller, J., Hazlett, V. (2019) Youth Engagement and Mental Health Preliminary Report. Canadian Journal of Public Health. 110:626-632. (Undergraduate Work)

Ramey, H.L.,  Lawford, H.L., Rose-Krashnor, L., Freeman, J., \textbf{Lanct\^{o}t, J.D.}. (2018) Engaging diverse Canadian youth in youth development programs: Program quality and community engagement. Children and Youth Services Review. 94: 20-26.  (Undergraduate Work)

% \textit{Other peer-reviewed contributions}\\
% Lastname, Firstname, Coauthor1, A. (In preparation) Universality in Betting Markets. Conference/Journal TBD. (PhD work)
% Lastname, Firstname*, Coauthor1, A. (2023) Network Defense. Undergraduate Thesis, University of Toronto. (Undergraduate work)

\textbf{c) Non-peer-reviewed contributions}

\vspace{-0.6em}
\textbf{Lanct\^{o}t, J.D.*}, Cornelius, S.P., (2024) Deep Learning for Network Attack and Defense. (Masters Thesis - To be published as an article instead.)

\textbf{Lanct\^{o}t, J.D.*}, Cornelius, S.P., (2022) Network Defense. (Undergraduate thesis)

\textbf{Lanct\^{o}t, J.D.*}, Cornelius, S.P. (2024) Complex Systems Day 2024. Oral presentation. (Masters work)

\textbf{Lanct\^{o}t, J.D.*}, Cornelius, S.P. (2024) 2024 CAP Congress. Oral presentation. (Masters work)

\textbf{Lanct\^{o}t, J.D.*}, Cornelius, S.P. (2023) Complex Systems Day 2023. Oral presentation. (Masters work)

Keanu, M.R*, \textbf{Lanct\^{o}t, J.D.}, Cornelius, S.P. (2023) APS March Meeting. Oral presentation. (Masters work)

\textbf{Lanct\^{o}t, J.D.*}, Cornelius, S.P. (2023) APS March Meeting 2023. Oral presentation. (Masters work)

Keanu, M.R*, \textbf{Lanct\^{o}t, J.D.}, Cornelius, S.P. (2022) APS March Meeting. Oral presentation. (Undergrad work)

\textbf{Lanct\^{o}t, J.D.*}, Cornelius, S.P. (2022) Complex Systems Day 2022. Oral presentation. (Undergrad work)

\textbf{Lanct\^{o}t, J.D.*}, Cornelius, S.P. (2021) CUPC 2021. Oral presentation. (Undergrad work)

\textbf{d) Technology transfer}

\vspace{-0.6em}
\textit{Next.js eCommerce (2023-Present):} Top contributor to an open-source eCommerce framework with over 5.1k stars on GitHub. (Professional development)

\textit{Scientific Computing Installation (2022-Present):} Developed open source configuration files for improving computational productivity in research for all lab members and peers. (Masters work)

\section*{II - Most significant contributions to research and development}
\vspace{-1.5em}
\textit{Deep Learning for Network Attack and Defense (Masters Thesis):} This research explored the capacity of Artificial Intelligence to learn features about networks, such as the Power Grid, and how hiding information about such networks from AI can limit network vulnerabilities. This work is significant as it laid the foundation for my current PhD research on network robustness. It demonstrated the potential of using AI techniques to attack critical infrastructure networks and how heuristic defenses were ineffective, but AI defenses do mitigate the performance of AI attackers -- yielding important implications for cybersecurity and network resilience. My role involved developing the AI models, conducting simulations, and analyzing the results. This work was entirely my own, conducted under the supervision of my advisor.

\textit{Mitochondrial network dynamics research (Accepted):} This research investigates how mitochondrial network structure impacts protein and molecule distribution. Using advanced simulation techniques, we've shown that well-connected and dynamically faster networks enhance particle spread, with branching networks formed through end-to-side fusion achieving optimal distribution. This work is significant as it provides insights into the fundamental mechanisms of cellular function and mitochondria network generation. As second author, my primary contributions include performing all of the network analysis of the networks generated by this novel model, as well as contributing manuscript preparation.

\textit{Next.js eCommerce open-source contribution :} As a top contributor to this popular eCommerce framework, I've played a crucial role in enhancing its functionality and stability. This project, with over 5.1k stars on GitHub, is significant as it simplifies the development of dynamic and user-friendly eCommerce websites, potentially impacting thousands of businesses and developers worldwide. My contributions include implementing new features, optimizing performance, and collaborating with a global community of developers. This work demonstrates my ability to apply theoretical knowledge to practical, real-world applications.

\section*{III - Applicant's statement}
\vspace{-1.5em}
\textbf{Research experience:} Throughout my academic career, I've developed strong skills in computational modeling, machine learning, and network analysis. My research has provided me with hands-on experience in applying AI techniques to complex network problems -- honing my ability to develop and implementing machine learning algorithms. My ongoing PhD research has further expanded my expertise in deep reinforcement learning, complex systems modeling, and data analysis. I've gained proficiency in using tools such as Python, PyTorch, Julia, and MATLAB for large-scale simulations and data processing.

% My work on mitochondrial network dynamics has enhanced my skills in analyzing computational models of complex biological systems.

\textbf{Relevant activities}\\
\textit{Awards and Distinctions:} I've received several prestigious awards, including the NSERC grant in 2023 (\$17,500) and 2021 (\$12,000), the Connections in Science award (\$1,000), and during undergrad was on the Dean's List for three consecutive years (2019-2022).

\textit{Academic Excellence:} During my graduate studies, I achieved a 4.25 cumulative grade point average (on a 4.33 scale). This result, coupled with my consistent placement on the Dean's List during undergrad, demonstrates my strong academic performance and dedication to my studies.

\textit{Presentations:} I've presented my research at multiple conferences, including the Complex Systems Day (2022-2024), APS March Meeting 2023, and CUPC 2021, showcasing my ability to communicate complex scientific ideas to diverse audiences.

\textit{Leadership:} As VP of Finance for PGSU in 2024, I've developed strong leadership and organizational skills, managing budgets and financial planning for a graduate student organization.

% \textit{Open Source Contributions:} My significant contributions to the Next.js eCommerce project and the Scientific Computing Installation repository demonstrate my ability to apply research skills to practical problems and my commitment to the broader scientific and developer communities.

\textit{Interdisciplinary Collaboration:} My work on various projects, from network security to mitochondrial dynamics, showcases my ability to work across disciplines and apply computational methods to diverse scientific problems.
\end{contributions}
    
    